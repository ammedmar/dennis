% !TEX root = ../dennis.tex

\section{Introduction} \label{s:introduction}

There is a tense trade-off in algebraic topology having roots reaching back to the beginning of its modern form.
This tension can be illustrated with the concept of cohomology.
The first approaches, dating back to Poincar\'e, are based on the subdivision of a space into simple contractible pieces.
These elementary shapes are made to generate a free graded module whose spatial relations define a differential used to compute cohomology.
This definition makes fairly clear certain geometric properties of cohomology, for example excision.
Yet, it is not easy to show that a continuous map of spaces induces a map between their associated cohomologies.
The functoriality just alluded to is trivial when defining cohomology in terms of homotopy classes of maps to Eilenberg--MacLane spaces, but the passage to the homotopy category erases geometric and combinatorial information and the resulting definition is not well suited for concretely presented spaces.

Cohomology as a graded abelian group is a fairly computable invariant but it has noticeable limitations, for example $\bC \rP^2$ and $S^2 \vee S^4$ are not distinguished by it.
Cohomology can be refined to a graded ring by endowing it with the cup product, an enhancement that does distinguishes these spaces.
In the spectral context, the product structure is defined through the wedge product of Eilenberg--MacLane spaces, whereas in the cellular setting it is obtained from a choice of cellular approximation to the diagonal map $X \to X \times X$.
Such cellular map induces a chain map
\begin{equation} \label{e:chain diagonal}
	\Delta \colon \gchains(X) \to \gchains(X) \ot \gchains(X)
\end{equation}
making the cellular chains of $X$ into a (differential graded) coalgebra.
The fact that the cup product on cohomology, induced by the linear dual of \eqref{e:chain diagonal}, is associative and (graded) commutative hints at the presence of additional structure extending the coalgebra structure on $\gchains(X)$.

In this survey article we will present, from a viewpoint that emphasizes their constructive nature, $C_\infty$ and $E_\infty$ extensions of $\Delta$ over the rationals and integers respectively.
The resulting algebraic structures control much of the homotopy theory of spaces.
For example, over the rationals, the quasi-isomorphism type of a $C_\infty$-coalgebra extension of the symmetrization of $\Delta$ determines the $\Q$-completion of $X$ under certain assumptions \cite{quillen1969rational, buijs2020liemodels}.
Whereas over the integers, and under similar assumptions, the quasi-isomorphism type of an $E_\infty$-coalgebra extension of $\Delta$ determines the homotopy type of $X$ \cite{mandell2006homotopy_type}.
%Both of these statements follow from the stronger fact that the relevant model categories are Quillen equivalent under certain assumptions.

\subsection*{Rational coefficients}

In \cref{s:rationally} we will study extensions of chain approximations to the diagonal with rational coefficients.
Over this field, a chain approximation to the diagonal can be symmetrized, giving rise to a cocommutative coalgebra.
This coalgebra cannot be made simultaneously coassociative, but this relation can be imposed in a derive sense through a family of coherent chain homotopies -- which also respect certain symmetry constrains -- and give rise to a so-called $C_\infty$-coalgebra structure.
One can think of $C_\infty$-coalgebras in terms of the somewhat more familiar notion of $A_\infty$-coalgebra where cocommutativity is satisfied strictly.
As a manifestation of Koszul duality, a $C_\infty$-coalgebra structure on cellular chains is equivalent to a differential on the completion of the free graded Lie algebra generated by the cells shifted downwards in degree by one.
This relates $C_\infty$-coalgebras to deformation theory, but we do not explore this deep connection here.
For cell complexes whose closed cells have the $\Q$-homology of a point, Dennis provided in \cite{sullivan2007appendix} a local inductive construction defining a $C_\infty$-coalgebra structure on their cellular chains.
We reprint a challenge he posted regarding the resulting structure.
\begin{displaycquote}[p.2]{lawrence2014interval}
	\textsc{Problem}. Study this free differential Lie algebra attached to a cell complex, determine its topological and geometric meaning as an intrinsic object.
	Give closed form formulae for the differential and for the induced maps associated to subdivisions.
\end{displaycquote}
As proven by Quillen, the quasi-isomorphism type of this $C_\infty$-coalgebra is a complete invariant of the rational homotopy type of simply-connected spaces.
For the $C_\infty$-coalgebra structure on the interval, Dennis and Ruth Lawrence addressed the challenge reprinted above introducing a formula for it which can be interpreted in terms of parallel transport of flat connections \cite{lawrence2014interval}, and for which the subdivision map is described by the Baker--Campbell--Hausdorff formula.

To generalize Quillen's equivalence of homotopy categories to one between (not necessarily 1-connected) simplicial sets, Buijs, F{\'e}lix, Murillo, and Tanr{\'e} \cite{buijs2020liemodels} extended to $n$-simplices the Lawrence--Sullivan structure building, constructively for $n \in \{2,3\}$ and inductively otherwise, $C_\infty$-coalgebra structures on their chains.
Their construction agrees after linear dualization with the one obtained by Cheng and Getzler in \cite{getzler2008transfering}, where they showed that the Kontsevich--Soibelman sum-over-trees formula defining the transfer of $A_\infty$-algebras through a chain contraction induces a transfer of $C_\infty$-algebras.
This allowed them to construct a $C_\infty$-algebra structure on simplicial cochains by transferring Dennis' polynomial differential forms through Dupont's contraction.
The resulting description is given in terms of rooted trees.

$C_\infty$-coalgebras are controlled by the operad $\com_\infty$ which is the Koszul resolution of the operad $\com$, i.e., the cobar construction applied to the $\lie$ cooperad, the Koszul dual cooperad of $\com$.
Another interesting resolution of $\com$ is constructed concatenating the bar and cobar constructions.
This resolution method is an algebraic version of the $W$-construction of Boardman--Vogt.
As Dennis and Scott Wilson considered, the resulting operad can be described using rooted trees with vertices colored black or white.
In \cite{vallette2020higherlietheory}, Daniel Robert-Nicoud and Bruno Vallette studied coalgebras over this resolution which they termed $CC_\infty$-coalgebras.
They constructed on the chain of standard simplices natural $CC_\infty$-coalgebra structures and described them explicitly using bicolored trees.

Despite some progress -- \cite{lawrence2019triangle, lawrence2021cells, buijs2019triangle}, \cite[\S6.5]{buijs2020liemodels} -- the ``closed form formulae'' part of the problem quoted before remains open.
One possible avenue to generalize to cubical chains the formula defining the Lawrence--Sullivan $C_\infty$-coalgebra on $\gchains(\gcube)$, is to define the tensor product of $C_\infty$-coalgebras and then extend it monoidally to all cubes via the isomorphism $\chains(\cube^n) \cong \gchains(\gcube)^{\ot n}$.
The monoidal structure on the category of $A_\infty$-coalgebras is defined through a chain approximation to the diagonal of the Stasheff polytopes compatible with the operad structure.
Unfortunately, the resulting $A_\infty$-coalgebra on $\gchains(\gcube)^{\ot 2}$ is not $C_\infty$.
This could be corrected through an algebraic symmetrization of the associahedral diagonal, but we do not pursue this here.

\subsection*{Integral coefficients}

In \cref{s:integrally} we will study extensions of chain approximations to the diagonal with integral coefficients.
In contrast to the situation over $\Q$, chain approximations to the diagonal over these coefficients cannot be taken to be symmetric with respect to transposition of tensor factors.
The resulting coalgebras can be made cocommutative and coassociative only up to coherent homotopies, that is to say, provided with the structure of a coalgebra over an $E_\infty$-operad.
The study of $E_\infty$-structures has a long history, where (co)homology operations \cite{steenrod1962cohomology, may1970general}, the recognition of infinite loop spaces \cite{boardman1973homotopy, may1972geometry}, and the complete algebraic representation of the $p$-adic homotopy category \cite{mandell2001padic} are key milestones.

Steenrod was the first to introduce homotopy coherent corrections to the broken symmetry of a chain approximation to the diagonal \cite{steenrod1947products}.
He did so on simplicial chains in the form of explicit formulae defining his cup-$i$ coproducts, with \mbox{cup-$0$} agreeing with the Alexander--Whitney chain approximation to the diagonal.
These coproducts are used to define Steenrod's mod $2$ cohomology operations and to effectively compute them in specific examples.

Extending the cup-$i$ coproducts of Steenrod, explicitly defined $E_\infty$-coalgebra structure on simplicial chains were introduced by McClure--Smith \cite{mcclure2003multivariable} and Berger--Fresse \cite{berger2004combinatorial}.
It turns out that this structure can be describe solely in terms of the Alexander--Whitney diagonal, the augmentation map and a chain version of the join of simplices \cite{medina2020prop1}.
This point of view can be abstracted using the language of props, which allows its application to other contexts, for example those defined by cubical chains \cite{medina2022cube_einfty} and the Adams' cobar construction \cite{medina2021cobar}.
We will review the resulting model of the $E_\infty$-operad, its action on simplicial and cubical chains, and explicit generalizations of the cup-$i$ coproducts to higher arities effectively constructing Steenrod operations at all primes \cite{medina2021may_st}.

We devote the final subsection to overview the use of cochain level structures in the classification of symmetry protected topological phases of matter.
