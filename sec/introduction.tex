
\section{Introduction} \label{s:introduction}

There is a tense trade-off in topology having roots reaching back to the beginning of its modern form.
This tension can be illustrated with the concept of cohomology.
The first approaches, dating back to Poincar\'e, are based on the idea of subdividing a space into simple contractible pieces.
These elementary shapes generate a free graded module and their spatial relations define the differential used to compute cohomology.
This definition makes certain geometric properties of cohomology, for example excision, fairly clear.
Yet, it is not easy to show that a continuous map of spaces induces a map between their associated cohomologies.
The functoriality just alluded to is trivial when defining cohomology in terms of homotopy classes of maps to Eilenberg--MacLane spaces, but this passage to the homotopy category forgets much geometric information and it is not easy to manipulate concretely for specific spaces.
The cohomology of spaces as a graded group is a fairly computable invariant but it has noticeable limitations.
For example, the spaces $\mathbb{C} P^2$ and $S^2 \vee S^4$ have isomorphic cohomology groups, which can be distinguished considering cohomology as a natural graded commutative ring.
In the spectral context, this finer structure is defined through the wedge product of Eilenberg--MacLane spaces, whereas in the cellular model of cohomology it follows from a choice of chain approximation to the diagonal map, i.e., a natural chain map
\[
\gchains(X) \to \gchains(X) \otimes \gchains(X)
\]
inducing a isomorphism when $X$ is a point.
In this survey article we will overview, from a viewpoint that emphasizes their constructive nature, some algebraic structures that arise from such chain diagonals and the homotopical information they encode.

\subsection*{Outline}

Our first goal will be to in \cref{s:rationally} study chain approximations to the diagonal with rational coefficients.
Over this field, a chain approximation to the diagonal can be chosen so that the permutational symmetries of the diagonal are preserved, resulting in a cocommutative coalgebra, which cannot be made simultaneously coassociative.
This relation can be imposed in derive sense through a family of coherent chain homotopies which also respect certain symmetry constrains and give rise to a so called $C_\infty$-coalgebra structure.
Through Koszul duality, this structure is equivalent to a differential on the completion of the free Lie algebra generated by the cells shifted in degree by $1$.
For cell complexes whose closed cells have the $\Q$-homology of a point, Dennis provided a local inductive construction defining such a structure \cite{sullivan2007appendix}.
As proven by Quillen, the quasi-isomorphism type of this $C_\infty$-coalgebra is a complete invariant of the rational homotopy type of simply-connected finite-type spaces.
To make this construction functorial and to generalize Quillen's equivalence of homotopy categories to one between general (not necessarily 1-connected) simplicial sets and complete dg Lie algebras, the authors of \cite{buijs2020liemodels} extended non\-/constructively the work of Dennis and Ruth Lawrence \cite{lawrence2014interval} by building a natural $C_\infty$-coalgebra structure on the chains of standard simplices.
The Sullivan-Lawrence $C_\infty$-coalgebra on the chains of the interval is given via an explicit formula with much geometric content.
The explicit nature of which has only been achieved for simplices of dimension less than or equal to $4$.

Our next goal will be to in \cref{s:integrally} study chain approximations to the diagonal map over the integers and finite fields.
In contrast to the situation over $\Q$, chain approximations to the diagonal over these coefficients cannot be taken to be symmetric with respect to the transposition of factors, and the resulting coalgebra can be made to be only cocommutative and coassociative up to coherent homotopies, that is to say, provided with an $E_\infty$-structure.
The study of these structures has a long history, where (co)homology operations \cite{steenrod1962cohomology, may1970general}, the recognition of infinite loop spaces \cite{boardman1973homotopy, may1972geometry}, and the complete algebraic representation of the $p$-adic homotopy category \cite{mandell2001padic} are key milestones.

Steenrod was the first to introduce homotopy coherent corrections to the lack of symmetry of a chain approximation to the diagonal \cite{steenrod1947products}.
He did so on simplicial chains in the form of explicit formulas defining his cup-$i$ coproducts, with \mbox{cup-$0$} the Alexander--Whitney chain approximation to the diagonal map.
These coproducts are used to define Steenrod's mod $2$ cohomology operations and to effectively compute them in specific examples.
For an explicit an explicit $E_\infty$-coalgebra structure on simplicial chains it was necessary to wait until the so called ``renaissance'' of operads.
Said structure was first introduced by McClure--Smith \cite{mcclure2003multivariable} and Berger--Fresse \cite{berger2004combinatorial} building on work by Steenrod and Benson.
It turns out that this structure can be describe solely in terms of the Alexander--Whitney diagonal, the augmentation map and a chain version of the join of simplices.

%Introducing such family of homotopies induce powerful invariants, including Steenrod operations on mod $p$ cohomology for every prime $p$.