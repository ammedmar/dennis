% !TEX root = ../dennis.tex

\section{\texorpdfstring{$C_\infty$}{C-infty}-coalgebras} \label{s:rationally}

Over the rationals, the problem of extending a chain approximation to the diagonal as a $C_\infty$-coalgebra is related to the study of Lie algebras.
In this section we recall this connection, and Dennis' construction of a $C_\infty$-coalgebra structure on the cellular chains of certain CW complexes.
We also discuss the resulting structure on the cellular chains of the interval, which is presented as a formula in the work of Dennis and Ruth Lawrence.
We discuss Quillen's functor from simplicial sets to complete dg Lie algebras, as extended by Buijs, F{\'e}lix, Murillo, and Tanr{\'e} through a cosimplicial $C_\infty$-coalgebra, and the problem of making this construction into explicit formulas extending the Sullivan--Lawrence interval.

\subsection{Quillen construction}

As a motivating example illustrating the connection between cocommutative and coassociative coalgebras and dg Lie algebras, let us recall the so-called \textit{Quillen construction}.
Consider one such coalgebra $C$, and form the free graded Lie algebra $L$ generated by the desuspension of $C$ regarded as a graded vector space.
Denote $\desus{1}C$ in $L$ by $L_1$, and by $L_2$ the linear span of brackets of elements in $\desus{1}C$.
The boundary map and coproduct respectively induce maps
\[
L_1 \xra{l_1} L_1,
\qquad
L_1 \xra{l_2} L_2
\]
of degree $-1$, and their relations ensure that $l_1 + l_2$ squares to $0$.
More explicitly,
\[
l_1(\desus{1}c) = -\desus{1} \bd c,
\qquad
l_2(\desus{1}c) = \frac{1}{2} \sum_i (-1)^{\bars{a_i}} \big[ \desus{1}a_i, \desus{1}b_i \big],
\]
where $\Delta(c) = \sum_i a_i \otimes b_i$.

The extension of $l_1 + l_2$ as a derivation of the Lie bracket makes $L$ into a free dg Lie algebra naturally associated to $C$.

\subsection{\texorpdfstring{$C_\infty$}{C-infinity}-coalgebras} \label{ss:c-infty definition}

The previous construction motivates the definition of $C_\infty$-coalgebras.
Before providing it, let us recall the notion of a \textit{complete chain complex} $(C, F)$, which is a filtered chain complex $C$
\[
C = F_0 C \supseteq F_1 C \supseteq \cdots
\]
such that
\[
C = \lim_{k\to \infty} C / F_k C.
\]
As expected, the \textit{completion} of a filtered chain complex $(C, F)$ is $\lim_{k\to \infty} C / F_k C$.

A $C_\infty$-\textit{coalgebra structure} on a graded vector space $C$ is the data of a differential on the completion, with respect to the filtration by number of brackets, of the free graded Lie algebra generated by $\desus{1}C$.

\subsection{$C_\infty$-coalgebras as commutative $A_\infty$-algebras} \label{ss:a-infty coalgebras}

We can interpret a $C_\infty$-coalgebra structure on $C$ in terms of the somewhat more familiar notion of $A_\infty$-coalgebra.

An $A_\infty$-\textit{coalgebra structure} on a graded vector space $C$ is a family of degree $k-2$ linear maps $\Delta_k \colon C \to C^{\ot k}$ satisfying for every $i \geq 1$ the following identity:
\begin{equation} \label{e:a-infty relations}
\sum_{k=1}^{i} \sum_{n=0}^{i-k} (-1)^{k+n+kn} \big( \id^{\ot i-k-n} \ot \Delta_k \ot \id^{\ot n} \big) \circ \Delta_{i-k+1} = 0.
\end{equation}
This is equivalent to the data of a differential on $\prod_{n \geq 1} (\desus{1} C)^{\otimes n}$, the augmentation kernel of the complete tensor algebra on the desuspension of $C$.
Indeed, such differential $d = \sum_{k \geq 1} d_k$ is determined by its restriction to $\desus{1} C$ with $d_k(\desus{1} C) \subset T^k(\desus{1} C)$, and the correspondence is explicitly given by
\[
\Delta_k = - s^{\ot k} \circ d_k \circ \desus{1},
\qquad
d_k = -(-1)^{\frac{k(k+1)}{2}} (\desus{1})^{\ot k} \circ \Delta_k \circ s.
\]
Notice that \eqref{e:a-infty relations} implies for any $A_\infty$-coalgebra that $\Delta_1$ squares to $0$, that $\Delta_2$ is a chain map with respect to $\Delta_1$, and that $\Delta_3$ is a chain homotopy between $(\Delta_2 \otimes \id) \circ \Delta_2$ and $(\id \ot \Delta_2) \circ \Delta_2$.

A $C_\infty$-coalgebra structure on a graded vector space $C$ is equivalent to an $A_\infty$-coalgebra structure on $C$ such that the image $d_k$ lies in the invariants of $(\desus{1} C)^{\ot k}$ under the graded action of $\Sym_k$, or, expressed in terms of the coproducts $\Delta_k$, one such that $\tau \circ \Delta_k = 0$, where
\[
\tau(c_1 \ot \dotsb \ot c_k) =
\sum_{i=1}^{k} \sum_{\sigma \in \Sym(i, k-i)}
\!\!\! \sign(\sigma) \,
(c_{\sigma(1)} \ot \dotsb \ot c_{\sigma(i)}) \ot
(c_{\sigma(i+1)} \ot \dotsb \ot c_{\sigma(k)})
\]
and $\Sym(i, k-i)$ denotes the set of $(i, k-i)$-shuffles.

\subsection{Sullivan's cellular $C_\infty$-coalgebra construction} \label{ss:dennis construction}

We now quote Dennis' inductive construction of a local $C_\infty$-coalgebra structure on the chains of cell complexes whose closed cells have the $\Q$-homology of a point \cite{sullivan2007appendix}.

Let $X$ be one such cellular complex and $L(X) = L$ be the free Lie algebra generated by the desuspension of its rational cellular chains $\desus{1}C$.

Let us start by choosing a chain approximation $\copr \colon C \to C \ot C$ to the diagonal, which we assume equivariant -- since we are working with rational coefficient -- and local, in the sense that $\copr(e_\alpha)$ is contained in the subcomplex generated by the tensor product of cells in the closure of $e_\alpha$.
We remark that $(C, \delta, \copr)$ is a cocommutative coalgebra which is in general not coassociative.
Let $\delta_1$ and $\delta_2$ be the respective maps from $L_1$ to $L_1$ and $L_2$ induced from $\bd$ and $\copr$.
We denote by the same symbols their extensions to $L$ as derivations.
We now quote Dennis' construction:

\begin{displaycquote}[p.251]{sullivan2007appendix}
	Interpreting the equation $\delta \circ \delta = 0$ as $[\delta, \delta] = 0$ where $[\cdot, \cdot]$ is the graded commutator.
	For any $\delta$ the Jacobi identity is $[\delta, [\delta, \delta]]$.
	Suppose $\delta^k = \delta_1 + \dots + \delta_k$ has been defined so that $[\delta^k, \delta^k]$ has the first nonzero term in monomial degree $k + 1$.
	Jacobi implies this error commutes with $\delta_1$; that is, it is a closed element in the complex $\Der(L)$ of derivations of $L$.
	If we work in the closure of a cell, the homology hypothesis implies that $\Der(L)$ has homology only in degrees $0$ and $1$.
	Therefore, the error, which lives in degree $2$, can be written as a commutator with $\delta_1$.
	Using the cells to generate a linear basis of each $L_k$ by bracketing, we choose this solution to lie in the image of the adjoint of $\delta_1$ to make it canonical.
	This canonical solution is $\delta_{k+1}$ and this completes the induction, since one knows at the beginning $\delta_1 \circ \delta_1$
	and $\delta_2$ is chain mapping; that is, $[\delta_2, \delta_1] = 0$.
\end{displaycquote}

Dennis' construction is such that $\delta e_\alpha$ is in the sub Lie algebra generated by the closure of the cells in $e_\alpha$, or, expressed in dual terms, the maps $\copr_r \colon C \to C^{\ot r}$ corresponding to the $\delta_r$ maps are local.

Recall Dennis' problem, quoted in the introduction, of determining the topological and geometric meaning of this $C_\infty$-coalgebra as an intrinsic object, and give closed form formulae for it and the induced maps associated to subdivisions.
We now quote the solution Dennis and Ruth Lawrence gave to this problem in the case of the interval.

\subsection{Lawrence--Sullivan interval} \label{ss:LS interval}

Let $L$ be a completed free graded Lie algebra with filtration given by number of brackets, and let $U(L)$ be the complete graded vector space of series on one indeterminate with values on $L$ whose filtration is induced from that of $L$, i.e., the $N^\th$-part of the filtration $U(L)$ contains series of the form
\[
\sum_{n=1}^\infty x_n t^n
\]
where $x_n$ is in $F_N L$ for every $n$.
Consider the linear operator given by
\[
\frac{d}{dt} \left(\sum x_n t^n\right) = \sum n \, x_n t^{n-1}
\]
and the formal differential equation
\[
\frac{du}{dt} = \bd v - \ad_v u
\]
where $\ad_v u = [v, u]$.
By formally solving this equation one defines the \textit{flow generated} by $v$ for any rational time $t_0$.

An element $u \in L$ is said to be \textit{flat} if it is in degree $-1$ and satisfies $\bd u = \frac{1}{2} [u,u]$.
It is common to refer to these as \textit{Maurer--Cartan} elements, but do not use this terminology.
We now quote Dennis and Ruth Lawrence theorem.
\begin{displaycquote}[Theorem 1]{lawrence2014interval}
	There is a unique completed free differential graded Lie algebra, $A$, with generating elements $a$, $b$ and $e$, in degrees -1, -1 and 0 respectively, for which $a$ and $b$ are flat while the flow generated by $e$ moves from $a$ to $b$ in unit time.
	The differential is specified by
	\[
	\bd e = \ad_e b + \sum_{i=0}^{\infty} \frac{B_i}{i!}(\ad_e)^i(b-a),
	\]
	where $B_i$ denotes the $i$th Bernoulli number defined as coefficients in the expansion
	\[
	\frac{x}{e^x-1} = \sum_{n=0}^{\infty} B_n \frac{x^n}{n!}.
	\]
\end{displaycquote}
We remark that Dennis conjectured an equivalence between the description above and the one obtained by applying his inductive procedure (\cref{ss:dennis construction}).
This conjecture was verified in \cite{parent2012interval} by Parent and Tanr\'{e}.

%As presented in \cite[p.77]{buijs2013algebraicmodels}.
%Let $C$ be the graded $\k$-module given by
%\[
%C_k = \begin{cases}
%\k\{y,z\} & k=0, \\
%\k\{c\} & k=1, \\
%0 & \text{otherwise},
%\end{cases}
%\]
%equipped with the collection $\Delta = \{\Delta_k \colon C \to C^{\ot k}\}_{k \geq 1}$ of linear maps defined by:
The Lawrence--Sullivan dg Lie algebra is describe in terms of the associated $C_\infty$-coalgebra by
\begin{align*}
& \Delta_1(c) = y-z, \quad
\Delta_1(y) = \Delta_1(z) = 0, \\
& \Delta_2(c) = -\frac{1}{2} \Big( c \ot (y+z) + (y+z) \ot c \Big), \quad
\Delta_2(y) = -y \ot y, \quad
\Delta_2(z) = -z \ot z, \\
& \Delta_k(c) = \sum_{p+q=k-1} \frac{B_{k-1}}{p!q!} c^{\ot p} \ot (y-z) \ot c^{\ot q}, \quad
\Delta_k(y) = \Delta_k(z) = 0, \quad k \geq 3.
\end{align*}

\subsection{Rational homotopy theory} \label{ss:cdgl model}

To algebraically model the rational homotopy category of spaces two models were introduced.
On one hand there is Dennis' commutative approach \cite{sullivan1977infinitesimal} based on an adjunction
\[
\begin{tikzcd}
\sSet \arrow[r, shift left=2pt, "A_{\mathrm{PL}}"] &
\cdga^\op \arrow[l, shift left=2pt, "\ \bars{\,\cdot\,}_S"]
\end{tikzcd}
\]
explained in detail in \cite{bibid} of these proceedings.
On the other, there is Quillen's Lie approach, introduced in \cite{quillen1969rational} and extended in \cite{buijs2013algebraicmodels, buijs2020liemodels}, which is based on an adjunction
\[
\begin{tikzcd}
\sSet \arrow[r, shift left=2pt, "\cL"] &
\cdgl \arrow[l, shift left=2pt, "\ \bars{\,\cdot\,}_Q"]
\end{tikzcd}
\]
where $\cdgl$ denotes the category of complete dg Lie algebras.
This adjunction is defined explicitly by
\[
\cL(X) = \colim_{\simplex^n \to X} \cL(\simplex^n), \qquad
\bars{L}_n = \cdga(\cL(\simplex^n), L),
\]
where $\cL(\simplex^\bullet)$ is the cosimplicial complete dg Lie algebra defined by a natural $C_\infty$-coalgebra structures on the cellular chains of standard simplices.

Using the principles presented in the previous subsection and a careful treatment of the simplicial structure, Buijs, F{\'e}lix, Murillo, and Tanr{\'e} \cite{buijs2020liemodels} introduced a construction of $\cL(\simplex^\bullet)$, which they characterize axiomatically by requiring that the generators associated to vertices are flat, and that the linear part is induced from the boundary of chains.

We mention that this structure is isomorphic to the one obtained by dualizing the simplicial $C_\infty$-algebra defined by the Homotopy Transfer Theorem of $C_\infty$-algebras applied to Dennis' polynomial differential forms and Dupont's contraction \cite{getzler2008transfering}.

The problem of finding explicit formulas for the $C_\infty$-coalgebra structure on the $n$-simplex remains open for $n > 3$.

\subsection{Operadic viewpoint}

The operad $C_\infty$ is defined as the cobar construction on the Lie cooperad, the Kozul dual cooperad of $\com$.
That is to say
\[
C_\infty = \cobar \lie^{\mathrm c}.
\]
The operad $C_\infty$ is a minimal projective resolution of $\com$.
A larger projective resolution is defined composing the bar and cobar constructions
\[
CC_\infty = \cobar \barconst \com.
\]
Since the bar and cobar constructions are defined in terms of rooted trees the $CC_\infty$ operad can be described using bicolored trees.
In \cite{vallette2020higherlietheory}, Robert-Nicoud and Vallette constructed a cosimplicial $CC_\infty$-coalgebra in terms of bicolored trees, and explored the induced adjuntion between simplicial sets and $L_\infty$-algebras.

An anecdote shared with the author by both Dennis and Bruno Vallete, is that this bicolored model brought them in contact for the first time; after a talk where Dennis used this pictorial description, Bruno, then a recent graduate, recognized it as the bar-cobar resolution of $\com$.