
\section{\texorpdfstring{$C_\infty$}{C-infty}-coalgebras} \label{s:rationally}



\subsection{Quillen construction}

Over the rationals, the problem of resolving the operad $\com$ is related to the study of Lie algebras.
As a motivating example, let us recall the \textit{Quillen construction} producing a dg Lie algebra from a commutative coassociative (dg) coalgebra $C$.

Let us form the free Lie algebra $L$ generated by the desuspension of $C$.
Denote $\desus{1}C$ in $L$ by $L_1$ and by $L_2$ the linear span of brackets of elements in $\desus{1}C$.
The boundary map and coproduct respectively induce maps
\[
L_1 \xra{l_1} L_1,
\qquad
L_1 \xra{l_2} L_2
\]
of degree $-1$, and their relations ensure that $l_1 + l_2$ squares to $0$.
More explicitly,
\[
l_1(\desus{1}c) = -\desus{1} \bd c,
\qquad
l_2(\desus{1}c) = \frac{1}{2} \sum_i (-1)^{\bars{a_i}} \big[ \desus{1}a_i, \desus{1}b_i \big],
\]
where $\Delta(c) = \sum_i a_i \otimes b_i$.

The extension of $l_1 + l_2$ as a derivation of the Lie bracket makes $L$ into a free differential graded Lie algebra, which is naturally associated to $C$.

\subsection{\texorpdfstring{$C_\infty$}{C-infinity}-coalgebras}

The previous construction can be extended and used to define $C_\infty$-coalgebras, which are coalgebras over a preferred resolution of $\com$ obtained using the theory of Koszul duality for operads.

First let us recall the notion of a \textit{complete chain complex} $(C, F)$, which is a filtered chain complex $C$
\[
C = F_0 C \supseteq F_1 C \supseteq \cdots
\]
such that
\[
C = \lim_{k\to \infty} C / F_k C.
\]
As expected, the \textit{completion} of a filtered chain complex $(C, F)$ is $\lim_{k\to \infty} C / F_k C$.

A $C_\infty$-\textit{coalgebra} structure on a graded vector space $C$ is the data of a differential on the completion, with respect to the filtration by number of brackets, of the free Lie algebra generated by $\desus{1}C$.

\subsection{$C_\infty$-coalgebras as commutative $A_\infty$-algebras}

We can interpret the structure of a $C_\infty$-structure on $C$ in terms of the somewhat more familiar notion of $A_\infty$-coalgebra.

An $A_\infty$-\textit{coalgebra structure} on a graded vector space $C$ is a family of degree $k-2$ linear maps $\Delta_k \colon C \to C^{\ot k}$ satisfying for every $i \geq 1$ that
\[
\sum_{k=1}^{i} \sum_{n=0}^{i-k} (-1)^{k+n+kn} \big( \id^{\ot i-k-n} \ot \Delta_k \ot \id^{\ot n} \big) \circ \Delta_{i-k+1} = 0.
\]
This is equivalent to the data of a differential on the augmentation kernel
\[
\prod_{n \geq 1} T^n(\desus{1} C)
\]
of the complete tensor algebra on the desuspension of $C$.
Indeed, such differential $d = \sum_{k \geq 1} d_k$ is determined by its restriction to $\desus{1} C$ with $d_k(\desus{1} C) \subset T^k(\desus{1} C)$, and the correspondence is explicitly given by
\[
\Delta_k = - s^{\ot k} \circ d_k \circ \desus{1},
\qquad
d_k = -(-1)^{\frac{k(k+1)}{2}} (\desus{1})^{\ot k} \circ \Delta_k \circ s.
\]

A $C_\infty$-coalgebra structure on a graded vector space $C$ is equivalent to an $A_\infty$-coalgebra structure on $C$ such that the image of the maps $d_k$ lie in the invariants of $T^k(\desus{1} C)$ under the graded action of $\Sym_k$, or, expressed in terms of the coproducts $\Delta_k$, one such that $\tau \circ \Delta_k = 0$ where
\[
\tau(c_1 \ot \dotsb \ot c_k) =
\sum_{i=1}^{k} \sum_{\sigma \in \Sym(i, k-i)} \sign(\sigma) \,
(c_{\sigma(1)} \ot \dotsb \ot c_{\sigma(i)}) \ot
(c_{\sigma(i+1)} \ot \dotsb \ot c_{\sigma(k)})
\]
and $\Sym(i, k-i)$ denotes the set of $(i, k-i)$-shuffles.

\subsection{Sullivan's cellular $C_\infty$-coalgebra construction} \label{ss:dennis construction}

We now quote Dennis' inductive construction of a local $C_\infty$-coalgebra structure on the chains of cell complexes whose closed cells have the $\Q$-homology of a point \cite{sullivan2007appendix}.

Let $X$ be one such cellular complex and $L(X) = L$ be the free Lie algebra generated by the desuspension of its rational cellular chains $\desus{1}C$.

The cellular approximation theorem allows us to choose a chain approximation $\copr \colon C \to C \ot C$ to the diagonal map $X \to X \times X$, which we assume equivariant -- since we are working with rational coefficient -- and local, in the sense that $\copr(e_\alpha)$ is contained in the subcomplex generated by the tensor product of cells in the closure of $e_\alpha$.
We remark that $(C, \delta, \copr)$ is a cocommutative coalgebra which is in general not coassociative.
Let $\delta_1$ and $\delta_2$ be the respective maps from $L_1$ to $L_1$ and $L_2$ induced from $\bd$ and $\copr$.
We denote by the same symbols their extensions to $L$ as derivations.
We now quote Dennis' construction:

\begin{displaycquote}[p.251]{sullivan2007appendix}
	Interpreting the equation $\delta \circ \delta = 0$ as $[\delta, \delta] = 0$ where $[\cdot, \cdot]$ is the graded commutator.
	For any $\delta$ the Jacobi identity is $[\delta, [\delta, \delta]]$.
	Suppose $\delta^k = \delta_1 + \dots + \delta_k$ has been defined so that $[\delta^k, \delta^k]$ has the first nonzero term in monomial degree $k + 1$.
	Jacobi implies this error commutes with $\delta_1$; that is, it is a closed element in the complex $\Der(L)$ of derivations of $L$.
	If we work in the closure of a cell, the homology hypothesis implies that $\Der(L)$ has homology only in degrees $0$ and $1$.
	Therefore, the error, which lives in degree $2$, can be written as a commutator with $\delta_1$.
	Using the cells to generate a linear basis of each $L_k$ by bracketing, we choose this solution to lie in the image of the adjoint of $\delta_1$ to make it canonical.
	This canonical solution is $\delta_{k+1}$ and this completes the induction, since one knows at the beginning $\delta_1 \circ \delta_1$
	and $\delta_2$ is chain mapping; that is, $[\delta_2, \delta_1] = 0$.
\end{displaycquote}

Dennis' construction is such that $\delta e_\alpha$ is in the sub Lie algebra generated by the closure of the cells in $e_\alpha$, or, expressed in dual terms, the maps $\copr_r \colon C \to C^{\ot r}$ corresponding to the $\delta_r$ maps are local.

\subsection{Rational homotopy theory}

To algebraically model the rational homotopy category of spaces two models were introduced.
On one side there is Dennis' commutative approach \cite{sullivan1977infinitesimal} based on an adjunction
\[
\begin{tikzcd}
\sSet \arrow[r, shift left=2pt, "A_{\mathrm{PL}}"] &
\cdga \arrow[l, shift left=2pt, "\ \bars{\,\cdot\,}_S"]
\end{tikzcd}
\]
explained in detail in \cite{bibid} of these proceedings.
On the other, there is Quillen's Lie approach introduced in \cite{quillen1969rational} and extended in \cite{buijs2013algebraicmodels, buijs2020liemodels} which is based on an adjunction
\[
\begin{tikzcd}
\sSet \arrow[r, shift left=2pt, "\cL"] &
\cdgl \arrow[l, shift left=2pt, "\ \bars{\,\cdot\,}_Q"]
\end{tikzcd}
\]
where $\cdgl$ denotes the category of complete differential graded Lie algebras.
This adjunction, given explicitly by
\[
\cL(X) = \colim_{\simplex^n \to X} \cL(\simplex^n), \qquad
\bars{L}_n = \cdga(\cL(\simplex^n), L),
\]
is defined by the construction of a natural $C_\infty$-coalgebra structures on the cellular chains of standard simplices, interpreted as a complete Lie algebra $\cL(\simplex^n)$.
This construction was accomplished by Buijs, F{\'e}lix, Murillo, and Tanr{\'e} in loc. cit. using the principles presented in the previous subsection together with a careful treatment of the simplicial structure.
The resulting $C_\infty$-coalgebras are explicitly described only for simplices of dimension $n \leq 3$.
The case $n = 1$, which plays an important role in their program, was studied earlier by Dennis and Ruth Lawrence \cite{lawrence2014interval} and we recall it next.

\subsection{Lawrence-Sullivan model for the interval}

\anibal{fill this in}