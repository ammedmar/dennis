
\section{\texorpdfstring{$C_\infty$}{C-infty}-coalgebras} \label{s:rationally}

\subsection{Quadratic and Koszul duality}

We start by recalling the duality that relates the operads $\com$ and $\lie$ as explained in \cite{ginzburg1994koszul}.
We refer to \cite[Ch.~7,~10]{loday2012operads} for a more detailed presentation.

Let $\com$ and $\lie$ be the operads controlling respectively commutative and Lie (co)algebras.
These are examples of \textit{quadratic operads} as we describe now.
An \textit{operadic quadratic data} is a pair $(E, R)$ consisting of an $\Sym$-module $E$ and a sub-$\Sym$-module $R$ of $\free(E)^{(2)}$, the sub-$\Sym$-module of the free operad generated by $E$ spanned by the composites of two elements.
Let $E^\vee$ be the linear dual of $E$ regarded as an $\Sym$-module with the sign representation, and let $R^\perp$ be the annihilator of $R$ in $\free(E^\vee)(3)$.
The \textit{quadratic dual operad} of $\free(E, R)$ is by definition the operad $\free(E^\vee, R^\perp)$ denoted $\free(E, R)^\shrik$.
The \textit{quadratic dual co-operad} $\free(E, R)^\antishrik$ is defined similarly as $\free^\c(s E^\vee, s^2 R^\perp)$ where $s$ denotes the the suspension of $\Sym$-modules.

It can be directly checked that
\begin{align*}
\com^\shrik &= \lie, \\
\lie^\shrik &= \com.
\end{align*}
Furthermore, if $\cobar$ denotes the cobar construction of co-operads, it can be shown that the canonical maps
\begin{align*}
\cobar \lie^\c \cong \cobar \com^\antishrik &\to \com, \\
\cobar \com^\c \cong \cobar \lie^\antishrik &\to \lie
\end{align*}
define minimal resolutions.
We say that a general quadratically presented operad $\cO$ is \textit{Koszul} if the map $\cobar \cO^\antishrik \to \cO$ is a quasi-isomorphism.
This implies, since $\cobar \cO^\antishrik$ is cofibrant, that $\cobar \cO^\antishrik$ is a resolution of $\cO$ or, in other words, a model for $\cO_\infty$.

Given a chain complex $C$, there is a natural bijection between $\cO_\infty$-coalgebra structures on it and homological differentials with vanishing linear term on the free complete $\cO^\antishrik$-algebra generated by $C$.
For example, $C_\infty$-coalgebras on $C$ can be identified with differentials on the completion of the free Lie algebra on $s^{-1} C$.

\subsection{$C_\infty$-coalgebras as commutative $A_\infty$-algebras}

Expanding on the last example.
Let $L$ be the completion of a free Lie algebra generated by a complex $s^{-1}C$.
Write $L$ as $L_0 \oplus L_1 \oplus L_2 \oplus \dotsb $ where $L_0$ is the ground field $\Q$, $L_1$ is $s^{-1}C$, $L_2$ is spanned by brackets of pairs of elements in $L_1$, etc.
Consider a differential $\delta$ on $L$ expanded into components $\delta = \delta_1 + \delta_2 + \dotsb$ where $\delta_k$ is determined by a degree $-1$ linear map $L_1 \to L_k$.

\anibal{Finish this...}

\subsection{Sullivan's cellular $C_\infty$-coalgebra construction} \label{ss:dennis construction}

We now quote Dennis' inductive construction of a local $C_\infty$-coalgebra structure on the chains of cell complexes whose closed cells have the $\Q$-homology of a point \cite{sullivan2007appendix}.

Let $X$ be one such cellular complex and $L(X) = L$ be the free Lie algebra generated by the desuspension of its rational cellular chains $s^{-1}C$.

The cellular approximation theorem allows us to choose a chain approximation $\copr \colon C \to C \ot C$ to the diagonal map $X \to X \times X$, which we assume equivariant, since we are working with rational coefficient, and local, in the sense that $\copr(e_\alpha)$ is contained in the subcomplex generated by the tensor product of cells in the closure of $e_\alpha$.
We remark that $(C, \delta, \copr)$ is a cocommutative coalgebra which is in general not coassociative.
Let $\delta_1$ and $\delta_2$ be the respective maps from $L_1$ to $L_1$ and $L_2$ induced from $\bd$ and $\copr$.
We denote by the same symbols their extensions to $L$ as derivations.
We now quote Dennis' construction:

\begin{displaycquote}[p.251]{sullivan2007appendix}
	Interpreting the equation $\delta \circ \delta = 0$ as $[\delta, \delta] = 0$ where $[\cdot, \cdot]$ is the graded commutator.
	For any $\delta$ the Jacobi identity is $[\delta, [\delta, \delta]]$.
	Suppose $\delta^k = \delta_1 + \dots + \delta_k$ has been defined so that $[\delta^k, \delta^k]$ has the first nonzero term in monomial degree $k + 1$.
	Jacobi implies this error commutes with $\delta_1$; that is, it is a closed element in the complex $\Der(L)$ of derivations of $L$.
	If we work in the closure of a cell, the homology hypothesis implies that $\Der(L)$ has homology only in degrees $0$ and $1$.
	Therefore, the error, which lives in degree $2$, can be written as a commutator with $\delta_1$.
	Using the cells to generate a linear basis of each $L_k$ by bracketing, we choose this solution to lie in the image of the adjoint of $\delta_1$ to make it canonical.
	This canonical solution is $\delta_{k+1}$ and this completes the induction, since one knows at the beginning $\delta_1 \circ \delta_1$
	and $\delta_2$ is chain mapping; that is, $[\delta_2, \delta_1] = 0$.
\end{displaycquote}

Dennis' construction is such that $\delta e_\alpha$ is in the sub Lie algebra generated by the closure of the cells in $e_\alpha$, or, expressed in dual terms, the maps $\copr_r \colon C \to C^{\ot r}$ corresponding to the $\delta_r$ maps are local.

\subsection{Rational homotopy theory}

To algebraically model the rational homotopy category of spaces two models were introduced.
On one side there is Dennis' commutative approach \cite{sullivan1977infinitesimal} based on an adjunction
\[
\begin{tikzcd}
\sSet \arrow[r, shift left=2pt, "A_{\mathrm{PL}}"] &
\cdga \arrow[l, shift left=2pt, "\ \bars{\,\cdot\,}_S"]
\end{tikzcd}
\]
explained in detail in \cite{bibid} of these proceedings.
On the other, there is Quillen's Lie approach introduced in \cite{quillen1969rational} and extended in \cite{buijs2013algebraicmodels, buijs2020liemodels} which is based on an adjunction
\[
\begin{tikzcd}
\sSet \arrow[r, shift left=2pt, "\cL"] &
\cdgl \arrow[l, shift left=2pt, "\ \bars{\,\cdot\,}_Q"]
\end{tikzcd}
\]
where $\cdgl$ denotes the category of complete differential graded Lie algebras.
This adjunction, given explicitly by
\[
\cL(X) = \colim_{\simplex^n \to X} \cL(\simplex^n), \qquad
\bars{L}_n = \cdga(\cL(\simplex^n), L),
\]
is defined by the construction of a natural $C_\infty$-coalgebra structures on the cellular chains of standard simplices, interpreted as a complete Lie algebra $\cL(\simplex^n)$.
This construction was accomplished by Buijs, F{\'e}lix, Murillo, and Tanr{\'e} in loc. cit. using the principles presented in the previous subsection together with a careful treatment of the simplicial structure.
The resulting $C_\infty$-coalgebras are explicitly described only for simplices of dimension $n \leq 3$.
The case $n = 1$, which we recall next, was studied earlier by Dennis and Ruth Lawrence \cite{lawrence2014interval}.

\subsection{Lawrence-Sullivan model for the interval}

\anibal{fill this in}