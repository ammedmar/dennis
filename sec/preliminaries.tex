
\section{Preliminaries} \label{s:preliminaries}

\subsection{Conventions}

Let $\k$ be a commutative and unital ring.
We work over the closed symmetric monoidal category of homologically graded differential graded $\k$-modules, referring to its objects and morphisms as chain complexes and chain maps respectively.

\subsection{Operads, cooperads, and props}

Let $\Sym$ be the category whose objects are the natural numbers and morphisms satisfy
\[
\Sym(m,n) = \begin{cases}
\Sym_n & n = m, \\
\hfil \emptyset & n \neq m,
\end{cases}
\]
where $\Sym_n$ is the $n^\th$ symmetric group.
The category of \textit{right $\Sym$-modules} is the functor category $\Fun(\Sym^\op, \Ch)$ and that of \textit{left $\Sym$-modules} is $\Fun(\Sym, \Ch)$.

For any chain complex $C$ we have the following important examples whose symmetric structure is given by transposition of factors:
\begin{equation} \label{e:endo and coendo operads}
\End_C(n) = \End(C^{\ot n}, C), \qquad
\coEnd_C(n) = \End(C, C^{\ot n}).
\end{equation}

\textit{Right operads} and \textit{cooperads} are, respectively, monoids and comonoids in the category of $\Sym$-modules when equipped with the following monoidal structure.
Given a right $\Sym$-module $M$, consider the following associated endofunctor of $\Ch$ defined on objects by
\[
C \mapsto \bigoplus_{r \geq 0} M(r) \ot_{\Sym_r} C^{\ot r}.
\]
The composition of these functors defines said monoidal structure.
The corresponding notions on left $\Sym$-modules are defined analogously.
Important examples of right and left operads are respectively given by $\End_C$ and $\coEnd_C$ for any chain complex $C$.
When $C = \k$ we denote either of these by $\com$.


\textit{Algebras} and \textit{coalgebras} over operads and cooperads are defined in terms of these monoids and comonoids respectively.
We will prioritize algebras over right operads and coalgebras over left cooperads.




\begin{gather*}
\biEnd_C(n) = \End(C^{\ot m}, C^{\ot n}).
\end{gather*}

\textit{Algebras} and \textit{coalgebras} over operads and cooperads are defined in terms of these monoids.





\subsection{Algebras and coalgebras over operads}


\subsection{The $\com$ operad}

