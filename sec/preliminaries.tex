
\section{Preliminaries} \label{s:preliminaries}

\subsection{Conventions}

Let $\k$ be a commutative and unital ring.
We work over the closed symmetric monoidal category of homologically graded differential graded $\k$-modules, referring to its objects and morphisms as chain complexes and chain maps respectively.

Through out this section $C$ will be used to denote a chain complex.

\subsection{Operads and props}

Let $\Sym$ be the category whose objects are the natural numbers and morphisms satisfy
\[
\Sym(m,n) = \begin{cases}
\Sym_n & n = m, \\
\hfil \emptyset & n \neq m,
\end{cases}
\]
where $\Sym_n$ is the $n^\th$ symmetric group.
The category of \textit{right $\Sym$-modules} is the functor category $\Fun(\Sym^\op, \Ch)$ and that of \textit{left $\Sym$-modules} is $\Fun(\Sym, \Ch)$.
We have the following important examples whose symmetric structure is given by transposition of factors:
\begin{equation} \label{e:endo and coendo operads}
\End_C(n) = \End(C^{\ot n}, C), \qquad
\coEnd_C(n) = \End(C, C^{\ot n}).
\end{equation}
A right $\Sym$-module $M$ induces an endofunctor of $\Ch$ defined on objects by
\[
C \mapsto \bigoplus_{r \geq 0} M(r) \ot_{\Sym_r} C^{\ot r}.
\]
A \textit{right operad} $\cO$ is a right $\Sym$-module with the structure of a monad on its associated endofunctor.
More explicitly, this corresponds to a collections of maps
\[
\begin{split}
\cO(r) \ot \cO(s_1) \ot \dots \ot \cO(s_r) & \to \cO(s_1 + \dots + s_r) \\
\k & \to \cO(1)
\end{split}
\]
satisfying certain associativity, unitality and equivariance relations.
The corresponding notions for left $\Sym$-modules are defined analogously.
Important examples of right and left operads are respectively given by $\End_C$ and $\coEnd_C$ with their composition structures.

These examples are generalized by the functor $\biEnd_C \colon \Sym^\op \times \Sym \to \Ch$ defined on objects by
\[
\biEnd_C(s,r) = \Hom(C^{\ot s}, C^{\ot r}),
\]
whose composition structure together with the tensoring of linear maps defines the paradigmatic example of a \textit{prop}.

An \textit{$\cO$-algebra} (resp. $\cO$-coalgebra) structure on $C$ where $\cO$ is a right (resp. left) operad is a morphism $\cO \to \End_C$ (resp. $\cO \to \coEnd_C$).
More explicitly, these respectively correspond to a compatible collection of chains maps
\[
\cO(r) \ot_{\Sym_r} C^{\ot r} \to C
\]
and of equivariant chain maps
\[
\cO(r) \ot C \to C^{\ot r}.
\]
Similarly, a \textit{$\cP$-bialgebra} structure on $C$ is a prop morphism $\cP \to \biEnd_C$, which corresponds to a collection of compatible chain maps
\[
\cP(s,r) \ot C^{\ot s} \to C^{\ot r}.
\]

The equivalence $\Sym \cong \Sym^\op$ induced by taking inverses can be used to interchange left and right $\Sym$-modules.
We will omit these adjectives when no confusion arises from doing so.

\subsection{The commutative operad and the cup product}

Of particular importance to us is the operad $\com$ given explicitly by $\com(n) = \k$ for every $n \geq 0$.
Notice that $\com$-algebras are the same as unital associative commutative algebras.

The motivating example of a $\com$-algebra for us is given by the cohomology of a cellular space $X$.
Any choice of a chain approximation to the diagonal
\begin{equation} \label{e:chain diagonal}
\gchains(X) \to \gchains(X)^{\otimes 2}
\end{equation}
makes its cohomology $\rH^\vee(X)$ into a $\com$-algebra via the \textit{cup product}
\[
\rH^\vee(X)^{\otimes 2} \to \rH^\vee(X)
\]
induced by the dualization of \eqref{e:chain diagonal}.
%\begin{equation*}
%\gchains^\vee(X)^{\otimes 2} \to \gchains^\vee(X),
%\end{equation*}
%where $\gchains^\vee(X) = \Hom(\gchains(X), \k)$.

We are interested in extending the cochain level product by lifting the cohomological structure.
To do so, we will consider certain cofibrant resolutions $\cO \to \com$ over different rings and construct natural explicit lifts
\[
\begin{tikzcd}
\cO \arrow[r, dashed] \arrow[d] &
\End_{C^\vee(X)} \arrow[d] \\
\com \arrow[r] &
\End_{H^\vee(X)}.
\end{tikzcd}
\]
We will also describe the homotopical properties encoded in these algebraic models of cellular spaces.

In the next section we study this problem over $\Q$ and in \cref{s:integrally} over $\Z$ and $\Fp$.