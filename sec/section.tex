
\newpage

\section{Section name}

\subsection{The Lawrence-Sullivan model}

As presented in \cite[p.77]{buijs2013algebraicmodels}.
Let $C$ be the graded $\k$-module given by
\[
C_k = \begin{cases}
\k\{y,z\} & k=0, \\
\k\{c\} & k=1, \\
0 & \text{otherwise},
\end{cases}
\]
equipped with the collection $\Delta = \{\Delta_k \colon C \to C^{\ot k}\}_{k \geq 1}$ of linear maps defined by:
\begin{align*}
& \Delta_1(c) = y-z, \qquad
\Delta_1(y) = \Delta_1(z) = 0, \\
& \Delta_2(c) = -\frac{1}{2} \Big( c \ot (y+z) + (y+z) \ot c \Big), \qquad
\Delta_2(y) = -y \ot y, \qquad
\Delta_2(z) = -z \ot z, \\
& \Delta_k(c) = \sum_{p+q=k-1} \frac{B_{k-1}}{p!q!} c^{\ot p} \ot (y-z) \ot c^{\ot q}, \qquad
\Delta_k(y) = \Delta_k(z) = 0, \qquad k \geq 3.
\end{align*}
These coproducts define an $A_\infty$-coalgebra structure on $C$, i.e., they satisfy:
\[
\sum_{k=1}^i \sum_{n=0}^{i-k} (-1)^{k+n+kn} \big( \id_C^{\ot i-k-n} \ot \Delta_k \ot \id_C^{\ot n} \big) \Delta_{i-k+1} = 0
\]
for every positive integer $i$.
We now check the first two cases:

\fullline

\noindent $i=1, k=1, n=0$:
\[
(-1)^1 \big( \id_C^{\ot 0} \ot \Delta_1 \ot \id_C^{\ot 0} \big) \Delta_{1}(c) =
- \Delta_1 \circ \Delta_1(c) = 0.
\]

\fullline

\noindent $i=2, k=1, n=0$:
\[
(-1)^1 \big( \id_C^{\ot 1} \ot \Delta_1 \ot \id_C^{\ot 0} \big) \Delta_{2}(c) =
\frac{1}{2}(y+z) \ot (y-z).
\]

\noindent $i=2, k=1, n=1$:
\[
(-1)^3 \big( \id_C^{\ot 0} \ot \Delta_1 \ot \id_C^{\ot 1} \big) \Delta_{2}(c) =
\frac{1}{2}(y-z) \ot (y+z).
\]

\noindent $i=2, k=2, n=0$:
\[
(-1)^2 \big( \id_C^{\ot 0} \ot \Delta_2 \ot \id_C^{\ot 0} \big) \Delta_{1}(c) =
- y \ot y + z \ot z.
\]

\fullline \ \par

\[
\boxed{\Delta_3(c) =
\frac{1}{12} \ (y-z) \ot c \ot c + \frac{1}{6} \ c \ot (y-z) \ot c + \frac{1}{12} \ c \ot c \ot (y-z)}
\]

\noindent $i=3, k=1, n=0$:
\begin{align*}
(-1)^1 \big( \id_C^{\ot 2} \ot \Delta_1 \ot \id_C^{\ot 0} \big) \Delta_3(c) & =
+ \frac{1}{12} \ (y-z) \ot c \ot (y-z) + \frac{1}{6} \ c \ot (y-z) \ot (y-z)
\end{align*}

\noindent $i=3, k=1, n=1$:
\begin{align*}
(-1)^3 \big( \id_C^{\ot 1} \ot \Delta_1 \ot \id_C^{\ot 1} \big) \Delta_3(c) =
- \frac{1}{12} \ (y-z) \ot (y-z) \ot c + \frac{1}{12} \ c \ot (y-z) \ot (y-z)
\end{align*}

\noindent $i=3, k=1, n=2$:
\begin{align*}
(-1)^5 \big( \id_C^{\ot 0} \ot \Delta_1 \ot \id_C^{\ot 2} \big) \Delta_3(c) =
-\frac{1}{6} \ (y-z) \ot (y-z) \ot c - \frac{1}{12} \ (y-z) \ot c \ot (y-z)
\end{align*}

\[
\boxed{\Delta_2(c) = -\frac{1}{2} \Big( c \ot (y+z) + (y+z) \ot c \Big)}
\]

\noindent $i=3, k=2, n=0$:
\begin{align*}
(-1)^2 \big( \id_C^{\ot 1} \ot \Delta_2 \ot \id_C^{\ot 0} \big) \Delta_2(c) &=
- \frac{1}{2} \Big( c \ot \Delta_2(y+z) + (y+z) \ot \Delta_2(c) \Big) \\ &=
+ \frac{1}{2} c \ot y \ot y + \frac{1}{2} c \ot z \ot z +
\frac{1}{4} \ (y+z) \ot c \ot (y+z) + \frac{1}{4} \ (y+z) \ot (y+z) \ot c
\end{align*}

\noindent $i=3, k=2, n=1$:
\begin{align*}
(-1)^5 \big( \id_C^{\ot 0} \ot \Delta_2 \ot \id_C^{\ot 1} \big) \Delta_2(c) &=
+ \frac{1}{2} \Big( \Delta_2(c) \ot (y+z) + \Delta_2(y+z) \ot c \Big) \\ &=
- \frac{1}{4} \ c \ot (y+z) \ot (y+z) - \frac{1}{4} \ (y+z) \ot c \ot (y+z)
- \frac{1}{2} y \ot y \ot c - \frac{1}{2} z \ot z \ot c
\end{align*}

\noindent $i=3, k=3, n=0$:
\begin{align*}
(-1)^3 \big( \id_C^{\ot 0} \ot \Delta_3 \ot \id_C^{\ot 0} \big) \Delta_1(c) = 0
\end{align*}

There are two pairs of summands that cancel immediately and two pairs that can be combined.
After multiplying by $12$ we get:
\begin{align*}
3 c \ot (y-z) \ot (y-z) \ - \
3 (y-z) \ot (y-z) \ot c \ + \ \\
3 (y+z) \ot (y+z) \ot c \ - \
3 c \ot (y+z) \ot (y+z) \\
6 c \ot y \ot y \ + \
6 c \ot z \ot z \ - \
6 y \ot y \ot c \ - \
6 z \ot z \ot c
\end{align*}

\anibal{works if the first line has the opposite sign which is implied if $\Delta_3$ does.}


\fullline \ \ \par

Let $\S(i,n-i)$ be the set of $(i,n-i)$-shuffles, permutations $\sigma \in \S_n$ such that
\[
\sigma(1) < \dots < \sigma(i) \quad \text{and} \quad
\sigma(i+1) < \dots < \sigma(n).
\]
We describe a permutation $\sigma \in \S_n$ by its image $(\sigma(1), \dots, \sigma(n))$

An $A_\infty$-coalgebra is a $C_\infty$-coalgebra iff it satisfies $\tau \circ \Delta_n = 0$ for every positive integer $n$, where, if $a = a_1 \ot \dots \ot a_n$:
\[
\tau(a) =
\sum_{i=1}^{n} \sum_{\sigma \in \S(i, n-i)} (-1)^{\sigma} \varepsilon_\sigma(a) (a_{\sigma(1)} \ot \dots \ot a_{\sigma(n)})
\]
and $\varepsilon_\sigma(a)$ Koszul sign associated to the action of $\sigma$.

The $A_\infty$-coalgebra $(C, \Delta)$ is a $C_\infty$-coalgebra.
We now check the first few cases of the identity $\tau \circ \Delta_n(c) = 0$:

\fullline

\noindent \boxed{$n=1$}
By convention, $\S(1,0)$ is empty so the claim follows from definition.

\fullline

\noindent \boxed{$n=2$}
\[
\Delta_2(c) = -\frac{1}{2} \Big( c \ot (y+z) + (y+z) \ot c \Big)
\]

$i=1, \sigma = (1,2)$:
\[
-\frac{1}{2} (-1)^1 \Big( c \ot (y+z) + (y+z) \ot c \Big)
\]

$i=1, \sigma = (2,1)$:
\[
-\frac{1}{2} (-1)^2 \Big( (y+z) \ot c + c \ot (y+z)\Big)
\]

\anibal{$i$ is not used in the exponent, but $\sigma$ is.}

\fullline

\noindent \boxed{$n=3$}
\[
\boxed{\Delta_3(c) =
\frac{1}{12} \ (y-z) \ot c \ot c + \frac{1}{6} \ c \ot (y-z) \ot c + \frac{1}{12} \ c \ot c \ot (y-z)}
\]

$i=1, \sigma = (1,2,3)$:
\[
- \frac{1}{12} \ (y-z) \ot c \ot c - \frac{1}{6} \ c \ot (y-z) \ot c - \frac{1}{12} \ c \ot c \ot (y-z)
\]

$i=1, \sigma = (2,1,3)$:
\[
\frac{1}{12} \ c \ot (y-z) \ot c + \frac{1}{6} \ (y-z) \ot c \ot c - \frac{1}{12} \ c \ot c \ot (y-z)
\]

$i=1, \sigma = (3,1,2)$:
\[
\frac{1}{12} \ c \ot (y-z) \ot c + \frac{1}{6} \ c \ot c \ot (y-z) - \frac{1}{12} \ (y-z) \ot c \ot c
\]

$i=2, \sigma = (1,2,3)$:
\[
\frac{1}{12} \ (y-z) \ot c \ot c + \frac{1}{6} \ c \ot (y-z) \ot c + \frac{1}{12} \ c \ot c \ot (y-z)
\]

$i=2, \sigma = (1,3,2)$:
\[
\frac{1}{12} \ (y-z) \ot c \ot c - \frac{1}{6} \ c \ot c \ot (y-z) - \frac{1}{12} \ c \ot (y-z) \ot c
\]

$i=2, \sigma = (2,3,1)$:
\[
\frac{1}{12} \ c \ot c \ot (y-z) - \frac{1}{6} \ (y-z) \ot c \ot c - \frac{1}{12} \ c \ot (y-z) \ot c
\]

\fullline

\newpage
\begin{align*}
\Delta_2(c \mid c) & =
- \frac{1}{2} \Big( c \ot (y+z) + (y+z) \ot c \Big) \mid c \\ &
- c \mid\frac{1}{2} \Big( c \ot (y+z) + (y+z) \ot c \Big)
\end{align*}