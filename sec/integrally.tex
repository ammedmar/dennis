
\section{\pdfEinfty-coalgebras} \label{s:integrally}


\subsection{Alexander--Whitney diagonal} \label{ss:aw diagonal}

The first chain approximation to the diagonal map was given in the simplicial context by \v{C}ech and Whitney building on independent work presented in the 1935 International Congress in Moscow by Alexander and Kolmogorov.
The original references are \cite{alexander1936ring, cech1936multiplication, whitney1938products} and a historical account is presented by Whitney in \cite[p.110]{whitney1988history}.
This chain map, referred to as the \textit{Alexander--Whitney diagonal}, is defined on canonical basis elements by the formula
\begin{equation} \label{e:alexander-whitney coalgebra}
\copr \big( [0,\dots,n] \big) = \sum_{i=0}^n \ [0, \dots, i] \ot [i, \dots, n].
\end{equation}
Together with the \textit{augmentation map}
\begin{equation} \label{e:augmentation map}
\aug \big( [0,\dots,n] \big) =
\begin{cases}
1 & n=0, \\ 0 & n=1,
\end{cases}
\end{equation}
this diagonal satisfies
\begin{gather}
\label{e:coassociativity relation}
(\copr \ot \id) \circ \copr = (\id \ot \copr) \circ \copr, \\
\label{e:counital relation}
(\aug \ot \id) \circ \copr = \id = (\id \ot \aug) \circ \copr,
\end{gather}
making the chains of any simplicial set $X$ into a natural coassociative counital coalgebra, which is referred to as the \textit{Alexander--Whitney coalgebra} of $X$.

We will use the following notation for general coalgebras
\begin{align*}
\copr^1 &= \copr, \\
\copr^k &= (\copr \ot \, \id) \circ \copr^{k-1}.
\end{align*}

\subsection{Steenrod cup-$i$ coproduct structure} \label{ss:cup-i}

As it can be seen directly from \eqref{e:alexander-whitney coalgebra}, the Alexander--Whitney coalgebra is not cocommutative.
In \cite{steenrod1947products}, Steenrod introduced coherent higher diagonals correcting homologically this lack of cocommutativity, and used them to define finer invariants on mod~2 cohomology, the celebrated Steenrod squares, whose effective construction we now review.

Let $C$ be a chain complex of $\Ftwo$-modules and regard $\Hom(C, C \ot C)$ as a chain complex of $\Ftwo[\Sym_2]$-modules.
Denote by the \textit{norm element} $1 + (12) \in \Ftwo[\Sym_2]$ by $N$.
A \textit{cup-$i$ coproduct structure} on $C$ is an equivariant chain map
$\cW(2) \to \Hom(C, C \ot C)$ where
\begin{equation} \label{e:minimal resolution r=2}
\begin{tikzcd} [column sep = .7cm]
\mathcal W(2) = \Ftwo[\Sym_2]\{e_0\} & \arrow[l, "N"'] \Ftwo[\Sym_2]\{e_1\} & \arrow[l, "N"'] \Ftwo[\Sym_2]\{e_2\} & \arrow[l, "N"'] \cdots
\end{tikzcd}
\end{equation}
is the minimal free resolution of $\Ftwo$ as a $\Ftwo[\Sym_2]$-module.
The image of $e_i$ is denoted by $\copr_i \colon C \to C \otimes C$ is referred to as the \textit{cup-$i$ coproduct}.

Let $C$ be equipped with a cup-$i$ coproduct structure.
The \textit{Steenrod square operations}
\[
Sq^k \colon \rH(C^\vee) \to \rH(C^\vee)
\]
on the homology of its dual chain complex $C^\vee = \Hom(C, \Ftwo)$ are defined for every integer $k$ by the formula
\begin{equation} \label{e:steenrod squares}
Sq^k \big( [\alpha] \big) = \big[ (\alpha \ot \alpha) \copr_{k - \bars{\alpha}}(-) \big]
\end{equation}
where brackets are used to denote represented elements in $\rH(C^\vee)$.

We will now review the natural cup-$i$ coproduct structure on simplicial chains introduced in \cite[p.293]{steenrod1947products} and reviewed in \cite[p.682]{mcclure2003multivariable}, see also  \cite{gonzalez-diaz1999steenrod, medina2021newformulas} for alternative descriptions.
The image of $[0, \dots, n]$ under $\copr_i$ is the sum over all sequences
\begin{equation*}
0 < p_1 < \cdots < p_{i+1} < n
\end{equation*}
of the tensor product element
\begin{equation*}
\begin{split}
[ 0, \dots, &{p_1} ] \ast [ {p_2}, \dots, {p_3} ] \ast \cdots \ast [ {p_{i+1}}, \dots, n ]\ \tensor \\
[ &{p_1}, \dots, {p_2} ] \ast \cdots \ast [ {p_{i}}, \dots, {p_{i+1}} ]
\end{split}
\end{equation*}
if $i$ is odd, and of
\begin{equation*}
\begin{split}
[ 0, \dots, &{p_1} ] \ast [ {p_2}, \dots, {p_3} ] \ast \cdots \ast [ {p_{i}}, \dots, {p_{i+1}} ]\ \tensor \\
[ &{p_1}, \dots, {p_2} ] \ast [ {p_3}, \dots, {p_4} ] \ast \cdots \ast [ {p_{i+1}}, \dots, n ]
\end{split}
\end{equation*}
if $i$ is even, where $\ast$ denotes the \textit{join} of simplices:
\begin{equation*}
[{p_{k-1}}, \dots, {p_{k}} ] \ast [ {p_{k+1}}, \dots, p_{k+2}] = [{p_{k-1}}, \dots, p_k, p_{k+1}, \dots, p_{k+2}].
\end{equation*}
In more succinct terms we have the following recursive definition:
\begin{align*}
& \copr_0 = \copr, \\
& \copr_i =
(\ast \ot \id) \circ (\id \ot (12)\copr_{i-1}) \circ \copr
\end{align*}
%\begin{align*}
%& \copr_0 = \copr, \\
%& \copr_i = \begin{cases}
%(\id \ot \ast) \circ (\copr_{i-1} \ot \id) \circ \copr & \text{ if } i \text{ is even}, \\
%(\ast \ot \id) \circ (23) \circ (\copr_{i-1} \ot \id) \circ \copr & \text{ if } i \text{ is odd}.
%\end{cases}
%\end{align*}
We will generalize these formulas in \cref{ss:e-infty generalization}.

\subsection{Join as a canonical homotopy}

As described above, all cup-$i$ coproducts can be expressed as compositions of the Alexander--Whitney diagonal and the join operation $\ast \colon \chains(\simplex^n)^{\ot 2} \to \chains(\simplex^n)$, which we now define with integral coefficients as
\begin{multline}
\ast \big(\left[v_0, \dots, v_p \right] \ot \left[v_{p+1}, \dots, v_q\right]\big) = \\
\begin{cases} (-1)^{p} \sign(\pi) \left[v_{\pi(0)}, \dots, v_{\pi(q)}\right] & \text{ if } v_i \neq v_j \text{ for } i \neq j, \\
\hfil 0 & \text{ if not}, \end{cases}
\end{multline}
where $\pi$ is the permutation that orders the vertices.

This is because the join and the Alexander--Whitney diagonal can be uses to construct chain homotopies (or higher degree boundaries between linear maps) canonically.
For example, given any two quasi-isomorphism $f, g \colon \chains(\simplex^n) \to \chains(\simplex^n)$ a canonical chain homotopy between them is given by
\[
H = (f \ast g) \circ \copr.
\]
To see this recall the augmentation map $\aug \colon \chains(\simplex^n) \to \k$ defined in \eqref{e:augmentation map} which is the counit of $\copr$, and notice that the join is a chain homotopy between $\aug \ot \, \id$ and $\id \ot \aug$, i.e.,
\begin{equation}
\partial \pr = \aug \ot \, \id - \id \ot \aug.
\end{equation}
Since $f$ and $g$ are quasi-isomorphisms we have $\aug \circ f = \aug \circ g = \aug$, so
\begin{align*}
\bd H &=
\big( \aug \ot \id - \id \ot \aug \big) \circ (f \ot g) \circ \copr \\ &=
\big(\aug \ot g - f \ot \aug \big) \circ \copr \\ &= g - f.
\end{align*}

\subsection{An \pdfEinfty-coalgebra on simplicial chains} \label{ss:e-infty generalization}

Cup-$i$ coproducts on simplicial chains are part of a so called $E_\infty$-coalgebra structure.
This is a natural coalgebra structure over an operad whose arity $r$ part is a chain complex of free $\k[\Sym_r]$-module with the $\k$-homology of a point.
Similar to Dennis' construction over $\Q$ of an $C_\infty$-coalgebra structure on cellular chains (\cref{ss:dennis construction}), the existence of an $E_\infty$-coalgebra structure over any coefficient ring can be guaranteed using an acyclic carrier argument \cite{eilenberg1953acyclic}.
The goal of this subsection is to describe explicitly an $E_\infty$-coalgebra structure on simplicial integral chains generalizing the construction of cup-$i$ coproducts of Steenrod (\cref{ss:cup-i}).

The collection of all linear maps $\chains(\simplex^n) \to \chains(\simplex^n)^{\ot r}$ for any $r$ that can be expressed as an arbitrary compositions of the Alexander--Whitney coproduct, the join product, and permutations of factors defines an $E_\infty$-coalgebra structure on the chains of standard simplices.
We remark that, since we are only considering maps whose domain is $\chains(\simplex^n)$, the join ---although used in their construction--- is not one of them.

The $E_\infty$-operad $\UM$ defining this structure can be abstracted from this example.
Roughly speaking, $\UM = \{\M(1,r)\}_{r \geq 0}$ is the operad associated to the prop $\M$ generated by symbols $\copr, \aug, \pr$ in biarities $(1,2)$, $(1,0)$, and $(2,1)$ of degree $0,0,1$ with $\bd \copr = 0$, $\bd \aug = 0$, and $\bd \pr = \aug \ot \, \id - \id \ot \aug$, modulo the relations $(\aug \ot \, \id) \circ \copr = \id = (\id \ot \aug) \circ \copr$ and $\aug \circ \, \ast = 0$.
In \cref{ss:homology of M} we review certain explicit chain contractions that can be used to compute the equivariant homology of $\UM$, and we use in \cref{ss:cup-(p,1) coproducts}.

Full details regarding the construction of the prop $\UM$ can be found in \cite{medina2020prop1} together with a comparison to the surjection operad \cite{mcclure2003multivariable}, a construction based on an earlier generalization of Steenrod's cup-$i$ coproducts \cite[\S4.5]{benson1998representations}.

\subsection{Cubical chains}

Let us consider the cellular chains on the interval $\gchains(\gcube)$ as a counital coalgebra in the usual way:
\begin{align*}
\copr[01] &= [0] \ot [01] + [01] \ot [1], &
\copr[0]  &= [0] \ot [0], &
\copr[1]  &= [1] \ot [1] \\
\aug[01] &= 0, &
\aug[0]  &= 1, &
\aug [1]  &= 1.
\end{align*}
This structure can be extended to the chains of cubical sets using the isomorphism
\[
\chains(\cube^n) \cong \gchains(\gcube)^{\ot n}
\]
and the fact that the tensor product of counital coalgebras receives this structure canonically as we now recall.
For $i \in \{1,2\}$ let $C_i$ be a counital coalgebra, the tensor product $C_1 \ot C_2$ has the same structure, defined by
\begin{align} \label{e:extension of coproduct}
\copr(c_1 \ot c_2) &= (23) \big( \copr(c_1) \ot \copr(c_2) \, \big), \\
\label{e:extension of augmentation}
\aug(c_1 \ot c_2) &= \aug(c_1) \aug(c_2),
\end{align}
where $(23)$ acts by permuting the tensor factors of $C_1 \ot C_1 \ot C_2 \ot C_2$.

For any cubical set $Y$ the induce structure on its chains agrees with that considered by Serre in \cite{serre1951homologie}, and we refer to it as the \textit{Serre coalgebra} of $Y$.

We can define an $E_\infty$-coalgebra structure extending the Serre coalgebra by describing an extension to all $\chains(\cube^n)$ of the map $\ast \colon \gchains(\gcube)^{\ot 2} \to \gchains(\gcube)$ defined to be non-zero only for
\[
\ast([0] \ot [1]) = [01], \qquad
\ast([1] \ot [0]) = - [01].
\]

For $i = 1,2$ let $A_i$ be a chain complex equipped with a degree $1$ map $\pr \colon A_i^{\ot 2} \to A_i$ and chain map $\aug \colon A_i \to \k$ such that $\aug \circ \pr = 0$ and $\partial \pr = \aug \ot \, \id - \id \ot \aug$, then the tensor product $A_1 \ot A_2$ has the same structure, explicitly defined by \eqref{e:extension of augmentation} and
\begin{equation} \label{e:extension of product}
\pr \big( (a_1 \ot a_2) \ot (a_1^\prime \ot a_2^\prime)\big) =
(\id \ot \aug \ot \ast \, + \, \pr \ot \aug \ot \, \id)(23)
\big( a_1 \ot a_2 \ot a_1^\prime \ot a_2^\prime \big)
\end{equation}
where the right-hand side is more explicitly given by
\[
(-1)^{\bars{a_2} \bars{a_1^\prime}} \big( \aug(a_1^\prime) \, a_1 \ot \ast (a_2 \otimes a_2^\prime)  \, + \, \pr(a_1 \otimes a_1^\prime) \ot \aug(a_2) \, a_2^\prime \big).
\]

Together, formulas \eqref{e:extension of coproduct}, \eqref{e:extension of augmentation}, and \eqref{e:extension of product} induce on the tensor product of $\M$-bialgebras the same structure.
In particular, $\chains(\cube^n) \cong \gchains(\gcube)^{\otimes n}$ is equipped with an $\M$-bialgebra structure induced from that in the cellular chain on the interval.
From it, a standard categorical construction -- a Kan extension along the Yoneda embedding~-- provides the chains on any cubical set with the structure of an $E_\infty$-coalgebra extending the Serre coalgebra structure or, more specifically, the structure of a $\UM$-coalgebra.

\subsection{Chain contractions for $\M$} \label{ss:homology of M}

We now explicitly describe chain contractions
\[
\begin{tikzcd}
\arrow[loop, distance=2em, in=195, out=165, "\sigma"'] \M(s,r) \arrow[r, shift left=3pt, "\proj"] &
\M(s,r-1) \arrow[l, shift left=3pt, "\incl"]
\end{tikzcd}
\]
for every $s \geq 1$ and $r \geq 0$.
In the next section we will use these to define cup-$(r, i)$ coproducts on $\UM$-coalgebras.

Recall that $\M$ is the prop generated by $\copr$, $\aug$, and $\pr$ modulo certain relations (\cref{ss:e-infty generalization}).
Let $\eta \in \M(s,r)$ be a composition of generators and let
\begin{align*}
\incl(\eta) & = (\id \ot \eta) \circ \copr, \\
\proj(\eta) & = (\aug \ot \, \id^{\ot r-1}) \circ \eta, \\
\sigma(\eta) & = (\pr \ot \, \id^{r-1}) \circ (\id \otimes \eta) \circ \copr.
\end{align*}
Using the relations defining $\M$ it is not hard to see that $\M(s,0) \cong \k$ so we have an explicit contraction
\[
\begin{tikzcd}
\arrow[loop, distance=2em, in=195, out=165, "h"'] \M(s,r) \arrow[r, shift left=3pt, "\proj^r"] &
\k \arrow[l, shift left=3pt, "\incl^r"]
\end{tikzcd}
\]
where
\begin{equation} \label{e:big chain contraction}
h = \sigma + \incl \sigma \proj + \dots + \incl^{r} \sigma \proj^{r}.
\end{equation}
In particular, given that by construction $\UM$ is a free $\Sym$-module, these chain contractions shows that $\UM$ is an $E_\infty$-operad.

\subsection{Steenrod cup-$(r,i)$ products} \label{ss:cup-(p,i) coproducts}

To generalize the notion of cup-$i$ coproduct structure, consider the minimal free resolution of $\k$ as a $\k[\Cyc_r]$-module
\begin{equation} \label{e:minimal resolution r>2}
\begin{tikzcd} [column sep = .5cm]
\mathcal W(r) = \k[\Cyc_r]\{e_0\} & \arrow[l, "\,T"'] \k[\Cyc_r]\{e_1\} & \arrow[l, "\,N"'] \k[\Cyc_r]\{e_2\} & \arrow[l, "\,T"'] \cdots.
\end{tikzcd}
\end{equation}
with
\begin{equation} \label{e:trnaposition and norm maps}
N = 1 + \rho + \dots + \rho^{r-1}, \qquad
T = \rho - 1.
\end{equation}

Let $C$ be a $\UM$-coalgebra, for example the chains on a simplicial or cubical set.
For $r \geq 2$, let $\psi(r) \colon \cW(r) \to \End(C, C^{\ot r})$ be the $\Cyc_r$-equivariant chain map defined recursively by
\begin{equation}\label{e:definition of psi}
\begin{split}
\psi(r)(e_0) & = \Delta^{r-1}, \\
\psi(r)(e_{2m+1}) & = h\,T\,\psi(r)(e_{2m}), \\
\psi(r)(e_{2m}) & = h\,N\,\psi(r)(e_{2m-1}),
\end{split}
\end{equation}
where $T$ and $N$ are explicitly defined in \eqref{e:trnaposition and norm maps} and $h$ in \eqref{e:big chain contraction}.
The \textit{Steenrod cup-}$(r, i)$ \textit{product} of $C$ is defined for every $r, i \geq 0$ as the image in $\End(C, C^{\ot r})$ of $\psi(e_i)$.

\subsection{Steenrod operations} \label{ss:steenrod operations}

Let $p$ be an odd prime.
We now present a construction, analogue to the one given in \eqref{e:steenrod squares} to define Steenrod square operations, defining Steenrod operations in mod $p$ cohomology.
We remark that Steenrod square operations are parameterized by the mod $2$ homology of $\Sym_2 = \Cyc_2$.
In the present context, any inclusion $\Cyc_p \to \Sym_p$ induces a surjection in mod $p$ homology.
Using the degrees where this surjection is not the zero map we have the following construction explained in more detain in \cite{steenrod1952reduced,steenrod1953cyclic,may1970general}.

Let $C$ be a $\UM$-coalgebra and $C^\vee = \Hom(C, \Fp)$.
We simplify notation and denote the explicit linear map $\psi(p)(e_i) \colon C \to C^{\ot p}$ defined in \eqref{e:definition of psi} simply by $\psi_i$.
For any integer $s$, the \textit{Steenrod operations}
\begin{equation*}
P_s \colon \rH(C^\vee; \Fp) \to \rH(C^\vee; \Fp)
\end{equation*}
and
\begin{equation*}
\beta P_s \colon \rH(C^\vee; \Fp) \to \rH(C^\vee; \Fp)
\end{equation*}
are defined by sending the class represented by a cycle $\alpha \in C^\vee$ of degree $q$ to the classes represented respectively for $\varepsilon \in \{0,1\}$ by
\begin{equation*}
\pm \, \alpha^{\ot p} \circ \psi_{(2s-q)(p-1)-\varepsilon}(-)
\end{equation*}
where a possible sign convention is motivated in \cite[(6.1)]{steenrod1953cyclic}.

